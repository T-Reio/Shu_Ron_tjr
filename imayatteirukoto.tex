\documentclass[dvipdfmx, 14pt]{jsarticle}
\usepackage{mathpazo}
\usepackage{amsmath,amssymb}
\usepackage{array}
\usepackage[hiresbb]{graphicx}
\usepackage{tikz}
\usepackage{textcomp}
\usepackage{dcolumn}
\usepackage{here}
\usepackage{lscape}
\usepackage[top=53truemm,bottom=80truemm,left=34truemm,right=18truemm]{geometry}
\begin{document}

博士課程後期への進学にあたる準備としては、前項で挙げた修士論文の制作が主たるものとなる。

Pope and Simonsohn (2011)は、アメリカのプロ野球リーグ (Major League Baseball, MLB) に所属する打者が、自らの打撃成績を表す指標の一つである打率が特定の値(.300)を上回ることに対し強いこだわりを持っていると考えられる行動を示すことを発見し、この値が打者にとっての参照点 (reference point) として機能している、と主張した。これに対し私の研究では、こうした行動が他の打撃指標でも観察されるかを確認した上で、彼らの契約内容に着目し、各の指標について一定の値を上回ることで、翌シーズンの報酬が不連続的に上昇するような金銭的インセンティブが存在するかどうかを検証した。

また、今後も継続して研究を行うために、分析対象となるデータの利用可能性の確認と収集を行っている。MLBに関するデータについては修士論文執筆の過程で収集可能な範囲を確認し、選手の成績、および選手ごとの詳細な契約内容について、十分なデータを取得することが可能であることをチェックした。また、契約条件に関してMLBとの比較を行うため、日本のプロリーグ(日本プロ野球機構)についても、契約に関するデータを取得するための活動を行っている。具体的には、2018年4月よりプロ球団のデータ入力業務に携わっており、球団からデータの提供を受けることを目指している。



\end{document}
