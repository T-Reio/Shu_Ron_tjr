\documentclass[dvipdfmx, 12pt]{jsarticle}
\usepackage{mathpazo}
\usepackage{amsmath,amssymb}
\usepackage{array}
\usepackage[hiresbb]{graphicx}
\usepackage{tikz}
\usepackage{textcomp}
\usepackage{dcolumn}
\usepackage{here}
\usepackage{lscape}
\usepackage[top=80truemm,bottom=26truemm,left=34truemm,right=18truemm]{geometry}
\begin{document}

私の研究者としての最大の目標は、経済学の知見を卑近な出来事・行動分析に応用し、人々に受け入れられやすい環境・構造を構築するためのアプローチを共有することにある。学問としての発展に寄与することはもちろん、その方法を日常に持ち込むことで、人々がより強く経済学のアプローチを意識するような伝え方を模索するのも、経済学者の重要な役割であると考える。

具体的な研究として、現在修士論文の題材として取り組んでいる、プロスポーツ、およびそのマネジメントにおける意思決定に対する分析がある。この研究では、アメリカのプロ野球リーグの成績と年俸のデータを用いて、選手の行動決定における行動経済学的な特性の存在、およびその頑健性を検証している。野球は個人の成績や、チームへの貢献度を様々な指標によって数値化することが可能であり、経済学の実証分析で用いられる数学的なアプローチを応用することが容易い。また、それらの指標の有効性は「セイバーメトリクス」と呼ばれる科学的な分析に基づく手法によって研鑽が続けられており、プレーヤーが意思決定を行う際のアウトカムを高い精度で表現するためのアプローチが継続的になされている。また選手と球団との間で結ばれる契約内容に関しても、特にアメリカの場合は詳細に公表されているため、金銭的なインセンティブと絡めた分析を行うことも可能である。一方で、こうしたデータを用いて経済学的な分析を行っている研究は数少なく、修士論文、また博士課程後期に進学後以降も継続して取り組むことが可能な領域である、と言える。

また、この研究と並行して、行動経済学・労働経済学を中心とした実証研究を行う。こちらは後期課程入学後にその内容を固めていくことになるが、草案としては、日本の生活保護制度の制度設計に関する分析を考えている。日本の医療扶助は保護受給者を通さずに直接医療従事者に対する支払を行う給付制度が採用されており、それによって起こる過剰診療のリスクや、その効果測定を行うことが考えられる。

\end{document}
