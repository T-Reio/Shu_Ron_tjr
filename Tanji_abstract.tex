\documentclass[dvipdfmx, 11pt]{jsarticle}
\usepackage{mathpazo}
\usepackage{amsmath,amssymb}
\usepackage{array}
\usepackage[hiresbb]{graphicx}
\usepackage{tikz}
\usepackage{textcomp}
\usepackage{dcolumn}
\usepackage{here}
\usepackage{lscape}
\usepackage[top=25truemm,bottom=30truemm,left=25truemm,right=25truemm]{geometry}
\begin{document}

\title{Reference Dependence and Monetary Incentive \\
-Evidence from Major League Baseball-}
\author{丹治伶峰 Reio Tanji \\
大阪大学大学院 経済学研究科 博士課程}
\date{}
\maketitle

Many empircal studies have revealed the existance of reference-point dependent preference in field settings. Some of them include cases from professional sports players' decision making. Their performance indexes are observed by the team manager, and then they offer the players contracts about the monetary rewards.

Then, if there exist some monetary incentives that encourage individuals to take behavior that ``appears'' to be reference dependent, then we should conclude that it is caused by the design of the contracts.

In this paper, we picked up a case of Major League Baseball (MLB) players. Pope and Simonsohn (2011) reported that MLB position players have reference-point dependent preferences about the number of their batting-average. They found there exists excess mass above .300 of player's batting-average (the number of base-hit devided by that of at-bat: chances to get a base-hit). However, this is not sufficient evidence of the reference-dependence of the players: there is the possibility of existance of monetary incentives.

Our contribution is to explore this possibility. First, we confirmed the evidences of excess mass in .300 of batting-average and other performance indexes, by McCrary (2008)'s manipulation test. Then, we tested if there existed any monetary incentives that encouraged the players to do so. Specifically, we examined whether the fixed part of their salary kink or notch at the possible reference points, using the method of local-linear regression.

We found three implications: first, there actually existed bunching in the players' batting-average and some of the other performance indexes. Second, surprisingly, there were little evidences that supported the monetary incentive hypothesis. We also found that there is no incentivised reward outside of the fixed part of payment. Third, among the variable indexes, .300 of batting-average was a particularly solid benchmarks for the players. We can apply our results to design more efficient contract in the situation of labor economics.

\end{document}
