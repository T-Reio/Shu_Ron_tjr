\documentclass[dvipdfmx,12pt]{beamer}

\usepackage{bxdpx-beamer}
\usepackage{pxjahyper}
\usepackage{minijs}
\usetheme{Annarbor}
\usepackage{mathpazo}
\usepackage{amsmath,amssymb}
\usepackage{graphicx}
\usepackage{array}

\title{Reference Point \\ for the Baseball Players in Japan}
\subtitle{Is the contribution of SABR metrics taken into account?}
\author{Reio TANJI}
\date{June.28,2018}
\institute{Osaka University}

\begin{document}
\begin{frame}\frametitle{Introduction}
\titlepage
\end{frame}

\section{Introduction}
\begin{frame}\frametitle{References}
 
 \begin{itemize}
  
  \item Pope \& Simonsohn (2011,Association for Phychological Science)
   
  ``Round Numbers as Goals : Evidence From Baseball, SAT Takers, and the Lab''
   
  \item Hakes \& Sauer (2006, Jornal of Economic Perspectives)
   
  ``An Economic Evaluation of the \textit{Moneyball} Hypothesis''
   
  \end{itemize}
 
\end{frame}

\begin{frame}\frametitle{Pope \& Simonsohn}

\begin{itemize}

\item Verify that \textbf{round numbers} in performance scales act as \textbf{reference points}, by examing three practical studies.

\item In the first study, they found that baseball players in MLB prefer finising the season with a batting average(AVG) just above .300, to that with just below .300.

\item Data : MLB player's play-by-play data from 1975 to 2008.

Players with at least 200 at bat (打数) : N=8,817

\end{itemize}

\end{frame}

\begin{frame}
\begin{center}

\includegraphics[width=12cm,height=5.25cm]{Pope_Simonsohn_F1.pdf}

\end{center}

 \begin{itemize}
 
 \item Players with .298 or .299 (0.97 \%) $<$ with .300 or .301 (2.30 \%), $Z=7.35$, $p<.001$.
 
 \item Control distribution : when 5 plate appearances left in the season.
 

 \end{itemize}

\end{frame}

\begin{frame}

\begin{center}

\includegraphics[width=10cm,height=5.75cm]{Pope_Simonsohn_F2A.pdf}

\end{center}

 \begin{itemize}
 
 \item Players with AVG of .299 was likely to get a base hit(43\%) than overall(22.8\%) at their last PA.
 
 $Z=3.62 , p <.001$.
 
 \end{itemize}

\end{frame}

\begin{frame}

\begin{center}

\includegraphics[width=10cm,height=5.75cm]{Pope_Simonsohn_F2B.pdf}

\end{center}

 \begin{itemize}
 
 \item .298 or .299 players tend to walk (四球) than .300 or .301 players.
 
 $Z=2.14$, $p=.032$.
 
 \end{itemize}

\end{frame}

\begin{frame}

\begin{center}

\includegraphics[width=9cm,height=5.75cm]{Pope_Simonsohn_F2C.pdf}

\end{center}

 \begin{itemize}
 
 \item If his AVG is just above .300, then he might end the season earlier by being substituted.
 
 $Z=8.29$ and $p<.001$.
 
 \end{itemize}

\end{frame}

\begin{frame}\frametitle{Pope \& Simonsohn}

 \begin{itemize}
 
 \item The behavior of baseball players proved the existence of the reference point of round numbers, such as batting average of .300.
 
 \item Limitations:
 
 There were only one relevant round number.
 
 Action to improve their performance took place on the last plate appearance.
 
 \end{itemize}

\end{frame}

\begin{frame}\frametitle{Hakes \& Sauer}

\begin{itemize}

\item \textit{``Moneyball Hypothesis''}

: Michael Lewis's claim that the valuation of skills in MLB player's market was grossly inefficient.

\end{itemize}

\end{frame}

\begin{frame}\frametitle{Extension}

 \begin{itemize}
 
 \item How about other batting indexes other than batting average?
 
 
 \end{itemize}

\end{frame}














\end{document}