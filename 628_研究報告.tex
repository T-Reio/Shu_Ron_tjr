\documentclass[dvipdfmx,12pt]{beamer}

\usepackage{bxdpx-beamer}
\usepackage{pxjahyper}
\usepackage{minijs}
\usetheme{Annarbor}
\usepackage{mathpazo}
\usepackage{amsmath,amssymb}
\usepackage{graphicx}
\usepackage{array}

\title{Reference Point \\ for the Baseball Players in Japan}
\subtitle{Is the contribution of SABR metrics taken into account?}
\author{Reio TANJI}
\date{June.28,2018}
\institute{Osaka University}

\begin{document}
\begin{frame}\frametitle{Introduction}
\titlepage
\end{frame}

\section{Introduction}
\begin{frame}\frametitle{References}

\begin{itemize}

\item Pope \& Simonsohn (2011,Association for Phychological Science)

``Round Numbers as Goals : Evidence From Baseball, SAT Takers, and the Lab''

\item Hakes \& Sauer (2006, Jornal of Economic Perspectives)

``An Economic Evaluation of the \textit{Moneyball} Hypothesis''

\end{itemize}

\end{frame}

\begin{frame}\frametitle{Pope \& Simonsohn}

\begin{itemize}

\item Verify that \textbf{round numbers} in performance scales act as \textbf{reference points}, by examing three practical studies.

\item In the first study, they found that baseball players in MLB prefer finising the season with a batting average(AVG) just above .300, to that with just below .300.

\end{itemize}

\end{frame}

\begin{frame}\frametitle{Hakes \& Sauer}

\begin{itemize}

\item \textit{``Moneyball Hypothesis''}

: Michael Lewis's claim that the valuation of skills in MLB player's market was grossly inefficient.

\end{itemize}

\end{frame}

\end{document}