\documentclass[dvipdfmx, 12pt]{article}
\usepackage{mathpazo}
\usepackage{amsmath,amssymb}
\usepackage{array}
\usepackage[hiresbb]{graphicx}
\usepackage{tikz}
\usepackage{textcomp}
\usepackage{dcolumn}
\usepackage{here}
\usepackage{lscape}
\usepackage[top=25truemm,bottom=30truemm,left=25truemm,right=25truemm]{geometry}
\begin{document}

\title{Reference Dependence and Monetary Incentive \\
-Evidence from Major League Baseball-}
\author{Reio Tanji}
\date{}
\maketitle

\leftskip = 25pt
\rightskip = 25pt
  \small
  \textit{
  Many empircal studies have revealed the existance of reference-point dependent preference in field settings, including cases from professional sports players' decision making, whose performance is observed by the manager and evaluated. If there is some incentives that leads the individuals take ``apparent'' reference point dependent behaviror, then we should rather think it is reference dependence of the managers that evaluates the players.. In this paper, I picked up the case of Major League Baseball players, which Pope and Simonsohn (2011) reported have reference-point dependent tastes about the number of their batting-average. I first confirmed the evidences that supports reference point dependent manipulation of batting-average and other performance indexes. Then I tested if there exist any monetary incentives that encourage the players to do so. My results are against this assumption: no overestimation of achieving the possible reference point was observed, which confirms that players adjust their aspiration level to reach their internal goal, even though there is no additional bonus that rewards their achievement.
  }



\leftskip = 0pt
\rightskip = 0pt
\normalsize


\section{Introduction}

Reference-point dependence is one of the most important concepts to evaluate outcome, and it affects the agents' following economic behavior. Classical economic models assume that economic agents evaluate choice/prospects according to abolute value of (expected) return. On the other hand, Tversky and Kahneman (1992) introduced behavioral assumption: reference-point dependent preference sets some target value of the outcome and then subjects regard the possible outcome as gain or loss from the target. For example, workers feel happy if her/his wage goes up to \$15 per hour, but vice versa if it goes down to \$15, although the absolute value of \$15 is actually same. In this case, s/he evaluates their new wage with the reference point of the previous one.

Prospect theory, which is advocated by Tversky and Kahneman, consists of two main charactaristics: one is probability weighting function, and the other is reference dependence, mentioned above. It enabled us to interpret cases that seems inconsistent with the traditional economic theory, and made it possible to understand them with some additional assumptions. Thus, a lot of following researches conducted in field settings or laboratories.

Reference dependence is also observed in the behavior of athletes. Pope and Schweizer (2011) found that professional golf players regard ``per,'' the standard number of shots determined according to the difficuly of each hole, as reference points. Also, Pope and Simonsohn (2011) tested the existance of the reference dependence in the Major League Baseball, a professional baseball league of America.  When dealing with the case of the professional athletes, however, we must pay attention to how their salary contract is designed.

Suppose the case of the professional golf player. Golf is essentialy competition of the total number of shots they needed to finish the whole tour, regardless of that of each single hole, or whether s/he saves per or not in the hole. Rank of order is determined according to this number, and those with better scores are rewarded. Then, what about when there is some monetary incentive to make effort to save per? That is, if every time s/he saved per in each hole, then s/he can get some additional bonus separated from their total score. In this case, then, making effort to save per can be interpreted as sufficiently ``rational'' choice, although the observed behavior itself appears to be evidence of the reference dependence.

In this paper, I picked up the evidence from Major League Baseball (MLB) position players' behavior following Pope and Simonsohn (2011), and considered the questions above. First, I specified the existance of some ``apparently'' possible behavior that shows reference point dependent preference of them. Then, I tested if this manipulation is in fact pursued from the player`s reference point dependent preference.

MLB position players seems to have some reference points, about their batting performance indexes: .300 of batting average is one of the possible ones. Pope and Simonsohn (2011) have shown, that there exists bunching just above .300 of the distribution of this index. As a required premise, I first tested if there is any evidence for the manipulation around the round numbers, not only .300 of batting average, both for other points of batting-average and other indexes, such as homerun or stolen-bases. The result supported the previous study and there observed similar tendency in some of other ones.

Confirming this, then I made examinations to answer the question, ``Is this observed manipulation truly driven by the reference dependence of the players?'' It is true that there exists bunching just above the possible reference points, but it is not sufficient yet, since team managers assign some incentives for the players to adjust their aspiration level to meet those points. I applied regression analysis using the data of the players` salary, and revealed that there does not exists such an incentive for them: Observed behavior is actually the reference dependence of the players. I also consulted other methodologies and evidences to confirm these ways also drew essentialy same results. Furtheremore, I submit other probability that is not confirmed due to lack of data, then conclude the paper.

This paper proceed as follows. In the Section 2, I review some literature and verify the standpoint of my paper. Section 3 presents theoretial framework and empirical way to specification, and make some conjecture. Section 4 describes the data I availed. Section 5 show the results of the analysis. Discussion about some alternative interpretation and non-statistical data are included in Section 6. Finally, Section 7 concludes the paper.

\section{Literature Review}

  Tversky and Kahneman (1992) mentioned reference point dependence as one of the two distinct respects of their prospect theory. The most primitive form of reference dependent utility function is:

   \[
  u(x | r) = \begin{cases}
  x - r & \text{ if }x \geq r \\
  \lambda (x - r) & \text{ if }x < r
\end{cases}
  \]
  where $x$ denotes a certain outcome, and $r$ is one of the reference points. This agent evaluates the outcome by the difference from the reference point. In adiition, they assume ``loss-aversion'' of the individual, or $\lambda > 1$. Those who have this type of utility function, than they regard same absolute amount of outcome in different way, depending on s/he faces gain or loss situation. ``Diminishing sensitivity,'' which is concave in facing gain and convex in facing loss is an advanced form of this specification.

  Diecidue and Van de Ven (2008)`s ``aspiration level'' model added discontinuity assumption: that is, a utility function that ``jumps'' at the reference point. When ther exists jump in their utility function, then individuals try to manipulate outcome level, paying addtional cost which was not accepted in the standard ``smooth'' form of utility function. As mentioned below, I exploit this assumption to my model in this paper.

  \begin{tabular}{ccc}
    \begin{minipage}[H]{0.3\textwidth}
      \begin{figure}[H]
        \begin{tikzpicture}
          [domain = -2:2, samples = 200, >= stealth]
          \draw[->] (-2,0) -- (2,0) node[right]{$x$};
          \draw[->] (0,-2) -- (0,2) node[above]{$u(x)$};
          \draw plot[domain = 0:1.7] (\x, \x);
          \draw plot[domain = -0.9:0] (\x, {2 * \x});
          \draw (0,0) node [below right] {$r$};
        \end{tikzpicture}
        \scriptsize
        \caption{primitive gain-loss function}
        \label{gain-loss}
      \end{figure}
    \end{minipage} &
    \begin{minipage}[H]{0.3\textwidth}
      \begin{figure}[H]
        \begin{tikzpicture}
          [domain = -2:2, samples = 200, >= stealth]
          \draw[->] (-2,0) -- (2,0) node[right]{$x$};
          \draw[->] (0,-2) -- (0,2) node[above]{$u(x)$};
          \draw plot[domain = 0:1.7] (\x, {sqrt( \x)});
          \draw plot[domain = -1.7:0] (\x, {-sqrt(2 * - \x)});
          \draw (0,0) node [below right] {$r$};
        \end{tikzpicture}
        \scriptsize
        \caption{diminishing sensitivity}
        \label{dim-sen}
      \end{figure}
    \end{minipage}&
    \begin{minipage}[H]{0.3\textwidth}
      \begin{figure}[H]
        \begin{tikzpicture}[domain = 0:4, samples = 200, >= stealth]
          \draw[->](-0.5, 0) -- (4.2, 0) node[right]{$x$};
          \draw[->](0, -0.5) -- (0, 3.7) node[above]{$u(x)$};
          \draw[-](2.2, -0.1) -- (2.2, 0.1);
          \draw[domain=0:2.2,samples=200,>=stealth] plot (\x, {sqrt(\x)});
          \draw[domain=2.2:4.1,samples=200,>=stealth] plot (\x, {sqrt(\x) + 0.8});
          \draw (0, 0) node[below left]{O};
          \draw (2.2, -0.3) node {$r$};
        \end{tikzpicture}
        \scriptsize
        \caption{jump at the reference point}
        \label{jump}
      \end{figure}
    \end{minipage}
  \end{tabular}

  \vspace{1zw}

  Individuals with such reference-point dependent utility try to adjust their effort level so as to achieve their internal target, or reference point. There are a number of empirical literature that specifies the existence of reference dependence in the field or lab studies. Farber(2008) applied this model to the labor supply of New York cab drivers to show that as soon as they reached daily target sales, they considers to quit, even when they reached it early in each day. Jones(2018) made analysis on the system of American tax payment. He showed that individuals try to manipulate their real payment by substituting it by donation or other charitable action, and that especially when facing losses, they make more effort. This observation is also caused loss-aversion, with the reference point of zero-payment threshold.

  Reference dependence also occurs in the cases of sports. One of the most well-known papers amoung them is Pope and Schweizer (2011). They obtained the data of professional golf players, to point out that in each hole, players behave as they take ``per'' as the reference point: Specifically, they scceeed their putts, significantly better when the putt was one to save per than when it was one to get ``eagle'' or ``birdie.'' Similarly, Allen \textit{et al} (2016) specified the existance of reference point dependence of marathon runners, using data about the finish time of enormous number of race in the United States. In this case, the distribution of finishing time has excess mass around evry hour and 30 minutes. Note that these cases are common in that even if they achieve their internal goals, they do not receive any monetary reward for their success. Professional golf players are awarded according to the total number of shots through the whole tour, not to the number of pers they saved.
  %注釈:サブ4で特別な表彰?

  Pope and Simonsohn (2011) mention a seemingly similar case. They picked up three empirical evidence of round number as reference points: SAT (a standardized test for college admission in the United States) scores, laboratory experiment, and baseball. In their section of baseball, they picked up the evidence of Major League Baseball (MLB) players to claim that they suggest their reference depenent preference by the distribution of their perfonmance index: batting-average (AVG). According to their paper, the position players (batters) pay attention to their batting-average (AVG), especially to finish each season with their batting average of just above .300. They obatined MLB season individual AVG data from 1975 to 2008 and observed position players (= players except for pitchers) with at least 200 at-bats in each season. Then, they found that their distribution of the batting-average has excess mass just above .300, which reveals the existance of manipulation there.Furthermore, they found that players with batting-average of just below .300 are more likely to hit a base-hit and less likely to get a base-on-balls. Both base-hits and base-on-balls avoid the batter from being gotten out, so for the team he belongs to, base-on-balls also have important value to win the game. However, batting-average does not count base-on-balls as the element to raise the number (For the definition of performance statistics, see Appendix), so they prefer getting hit to base-on-balls. Thus, observed behavior they claims is sufficient evidence that shows the existance of round-number reference point dependent preference of the MLB players.

  It is true that there is observed behaviror similar to the cases of Pope and Schweizer (2011) or Allen \textit{et al} (2016). However, one important thing we have to take care of is there exists procedure of contract between the player and the team manager: those who evaluate the player. In other words, it may owe to their monetary value function, not by the preference of themselves. This is the main contribution of my research.

  Pope and Simonsohn stated in their own paper that they conducted analysis only for batting-average, and following research is to be made. So I first follow this: ``Is batting-average unique case?'' Then, I test if there exists monetary incentive for the player. If the team managaers they belong to adopt a system of salary that discontinuously ``jump'' by a certain performance index reaching to the point where manpulation is observed, then for the players it can be interpreted as rational choice, given the contract design. In general, players with their performance index just above these cutoff point and those just below the point have almost same ability as a baseball player. At least, it is natural to think there is no reason to treat players discontinuously better, only because he achieve the cutoff. Then, it is interpreted that it is rather the team manager than the players themselves who have the reference-point dependent preference, which makes the players encouraged to meet their goals.

  On the other hand, if there exists no evidence that team managers evaluate the players by the achievement of the cutoff, then we can say that the observed behavior is truly drawn by their own reference dependence. In addition, the consistency is so strong that even there exists no rational reason, they try to reach there. Analysing this and verify which hypothesis is my main contribution of this paper.

\end{document}
