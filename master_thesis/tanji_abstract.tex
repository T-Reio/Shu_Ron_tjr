\documentclass[dvipdfmx, 12pt]{article}
\usepackage{mathpazo}
\usepackage{amsmath,amssymb}
\usepackage{array}
\usepackage[hiresbb]{graphicx}
\usepackage{tikz}
\usepackage{textcomp}
\usepackage{dcolumn}
\usepackage{here}
\usepackage{lscape}
\usepackage[top=25truemm,bottom=30truemm,left=25truemm,right=25truemm]{geometry}
\begin{document}


\title{Reference Dependence and Monetary Incentive \\
-Evidence from Major League Baseball-}
\author{Reio Tanji}
\date{}
\maketitle

Many empircal studies have revealed the existance of reference-point dependent preference in field settings. Some of them include cases from professional sports players' decision making. Their performance indexes are observed by the team manager, and then they offer the players contracts about the monetary rewards. Then, if there exist some monetary incentives that encourage individuals to take behavior that ``appears'' to be reference dependent, then we should conclude that it is caused by the design of the contracts. In this paper, we picked up a case of Major League Baseball players, whom Pope and Simonsohn (2011) reported have reference-point dependent preferences about the number of their batting-average. We confirmed the evidences of excess mass in .300 of batting-average and other performance indexes, and tested if there existed any monetary incentives that encouraged the players to do so. We found three implications: first, there actually existed bunching in the players' batting-average and some of the other performance indexes. Second, surprisingly, there were little evidences that supported the monetary-incentive hypothesis. Third, among the variable indexes, .300 of batting-average was a particularly solid benchmarks for the players.

\end{document}
