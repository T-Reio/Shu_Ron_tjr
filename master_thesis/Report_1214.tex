\documentclass[dvipdfmx,12pt]{beamer}

\usepackage{bxdpx-beamer}
\usepackage{pxjahyper}
\usepackage{minijs}
\usetheme{AnnArbor}
\usepackage{mathpazo}
\usepackage{amsmath,amssymb}
\usepackage{graphicx}
\usepackage{array}
\usepackage{tikz}
\usepackage{wrapfig}
\usepackage{float}
\usepackage{here}
\usepackage{lscape}
\setbeamertemplate{navigation symbols}{}

\title{Reference Dependence and Monetary Incentive}
\subtitle{-Evidence from Major League Baseball-}
\author{Reio TANJI}
\date{Dec 14th, 2018}
\institute{Osaka University}

\begin{document}

\begin{frame}\frametitle{}
\titlepage
\end{frame}

\section{Introduction}
\begin{frame}\frametitle{Abstract}
  \begin{itemize}
    \item Empirical research that specifies the existance of reference point dependence observed in field setting:

    We pick up evidence of Major League Baseball (MLB)

    \item Players take some round numbers of the batting performance indexes as reference points, and adjust their effort level to meet the goals

    \item There are NOT observed any evidence for the monetary incentives that is paid to the players if they achieve these internal goals
  \end{itemize}
\end{frame}

\begin{frame}\frametitle{}

\end{frame}

\begin{frame}\frametitle{}

\end{frame}

\begin{frame}\frametitle{}

\end{frame}

\begin{frame}\frametitle{}

\end{frame}

\begin{frame}\frametitle{}

\end{frame}

\begin{frame}\frametitle{}

\end{frame}

\begin{frame}\frametitle{}

\end{frame}

\begin{frame}\frametitle{}

\end{frame}

\begin{frame}\frametitle{}

\end{frame}

\begin{frame}\frametitle{}

\end{frame}

\begin{frame}\frametitle{}

\end{frame}

\begin{frame}\frametitle{}

\end{frame}

\begin{frame}\frametitle{}

\end{frame}

\begin{frame}\frametitle{}

\end{frame}

\end{document}
