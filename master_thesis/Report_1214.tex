\documentclass[dvipdfmx,12pt]{beamer}

\usepackage{bxdpx-beamer}
\usepackage{pxjahyper}
\usepackage{minijs}
\usetheme{AnnArbor}
\usepackage{mathpazo}
\usepackage{amsmath,amssymb}
\usepackage{graphicx}
\usepackage{array}
\usepackage{tikz}
\usepackage{wrapfig}
\usepackage{float}
\usepackage{here}
\usepackage{lscape}
\setbeamertemplate{navigation symbols}{}

\title{Reference Dependence and Monetary Incentive}
\subtitle{-Evidence from Major League Baseball-}
\author{Reio TANJI}
\date{Dec 14th, 2018}
\institute{Osaka University}

\begin{document}

\begin{frame}\frametitle{}
\titlepage
\end{frame}

\section{Introduction}
\begin{frame}\frametitle{Abstract}
  \begin{itemize}
    \item Empirical research that specifies the existance of reference point dependence observed in field setting:

    We pick up evidence of Major League Baseball (MLB)

    \item Players take some round numbers of the batting performance indexes as reference points, and adjust their effort level to meet the goals

    \item There are NOT observed any evidence for the monetary incentives that is paid to the players if they achieve these internal goals
  \end{itemize}
\end{frame}

\begin{frame}\frametitle{Introduction}
  \begin{itemize}
    \item Reference dependence is one of the two main charactaristics of the Tversky and Kahneman (1992)'s prospect theory:

    Individuals evaluate outcomes by the relative value to their internal benchmarks, or reference point, not by their absolute ones.

    \item Prospect theory enabled us to interpret some inconsistent empirical decision making with the traditional microeconomic theory, by applying additional assumptions.

    \item There are a lot of following researches that tests the reference dependence in field or laboratory settings.
  \end{itemize}
\end{frame}

\begin{frame}\frametitle{Literature}
  Pope and Simonsohn (2011)
  \begin{itemize}
    \item presents three empirical evidences that verify the reference dependence, with the reference points ``round numbers.''

    \item One of them picked up Major League Baseball (MLB) players, about the observed attitude to their performance indexes.

    \item MLB position players manipulate their batting-average (AVG), in order to meet their internal goals: .300

    \item As a results, there is observed excess mass, or ``bunching'' around .300 of AVG.
  \end{itemize}
\end{frame}

\begin{frame}\frametitle{Contribution}
\begin{itemize}
  \item Professional athletes receive monetary rewards according to the contracts they signed.

  \item Their contracts might include some incentivesed parts, which pay them additional bonus when their AVG reaches a certain cutoff point.

  \item If so, the observed behavior might be caused by the discontinuity of their profit function, not by the reference dependence.

  \item The contribution of our research is to examine this: examine if there exists any monetary incentives that make players make effort to the cutoff point.
\end{itemize}
\end{frame}

\section{Frameworks and Empirical Methods}
\begin{frame}\frametitle{Theoretical Frameworks}
  \begin{itemize}
    \item Following
  \end{itemize}
\end{frame}

\begin{frame}\frametitle{Specification: Manipulation}

\end{frame}

\begin{frame}\frametitle{Specification: Contract Design}

\end{frame}

\begin{frame}\frametitle{Data}

\end{frame}

\section{Results}
\begin{frame}\frametitle{Results: Manipulation}

\end{frame}

\begin{frame}\frametitle{}

\end{frame}

\begin{frame}\frametitle{Results: Contract Design}

\end{frame}

\begin{frame}\frametitle{}

\end{frame}

\begin{frame}\frametitle{Summary}

\end{frame}

\begin{frame}\frametitle{Considering Alternative Explanations}

\end{frame}

\section{Conclusions}
\begin{frame}
  \begin{itemize}
    \item 
  \end{itemize}
\end{frame}

\begin{frame}\frametitle{Reference}
  \footnotesize
  \begin{itemize}
    \item
  \end{itemize}
\end{frame}

\end{document}
