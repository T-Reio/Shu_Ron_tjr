\documentclass[dvipdfmx]{article}
\usepackage{mathpazo}
\usepackage{amsmath,amssymb}
\usepackage{array}
\usepackage[hiresbb]{graphicx}
\usepackage{tikz}
\usepackage{textcomp}
\usepackage{dcolumn}
\begin{document}

\title{Reference Point for Professional Athletes}
\author{Reio Tanji}
\date{}
\maketitle

\large

\section{Introduction}



In this paper, I specified if such behavior is in fact drived by
reference-dependent preference. In other words, I try to answer
a question ``Are there any monetary incentives that make players
turn to .300 of batting-average? ``

\section{Theoretical Frameworks and Literature Review}



\section{Method and Data}

 \subsection{Empirical Method}

 \subsection{Data Description}





\section{Result}



\section{Impact of `` \textit{Moneyball} ``
- Time Series Analysis -}

Through the previous section, I conducted analysis including
all the sample as same, as Pope and Simonsohn (2011) did.
However, there is some time-specific factor that affect on
the behavior of the players, such as the relationship
between the player and the team manager, what index
was of the most importance to evaluate players, or some
macroeconomic trend.


\section{Discussions and Conclusion}




\end{document}
