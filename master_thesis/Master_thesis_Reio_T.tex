\documentclass[dvipdfmx, 12pt]{article}
\usepackage{mathpazo}
\usepackage{amsmath,amssymb}
\usepackage{array}
\usepackage[hiresbb]{graphicx}
\usepackage{tikz}
\usepackage{textcomp}
\usepackage{dcolumn}
\usepackage{here}
\usepackage{lscape}
\usepackage[top=30truemm,bottom=30truemm,left=25truemm,right=25truemm]{geometry}
\begin{document}

\title{Reference Dependence and Monetary Incentive \\
-Evidence from Major League Baseball-}
\author{Reio Tanji}
\date{}
\maketitle

\begin{center}
  (Abstract)
\end{center}


\section{Introduction}

Reference-point dependence is one of the important concepts to
evaluate outcome, and it affects the agents' following
economic behavior. Classical economic models assume that
economic agents evaluate choice/prospects according to abolute
value of (expected) return. On the other hand, Tversky and
Kahneman (1992) introduced behavioral
assumption: reference-point dependent preference sets some
target value of the outcome and then subjects regard the
possible outcome as gain or loss from the target.
For example, workers feel happy if her/his wage goes up
to \$15 per hour, but vice versa if it goes down to \$15,
although the absolute value of \$15 is actually same. In this case,
s/he evaluates their new wage with the reference point of
the previous one.

When we pick up the evidence from competitive or incentivised
situation or the behavior that is monitored by other people
and evaluated depending on the performance they produced,
however, we have to pay attention  to the design of the contract.
If there is some ``rational''
motivation to make effort to achieve their goals,
then we can think that observed
choice is evidence of the existance of behavioral preference of
the principals, rather than the agents.

In this paper, I picked up the evidence from Major League Baseball
(MLB) players' behavior and find some ``possible'' choice that
shows reference point dependent preference of them.
Concretely, MLB batters seems to have a reference point of
a batting statistics, .300 of batting average, and make effort
to achieve it. This tendency was not observed other batting statistics,
even though those that occupy more weight in evaluating players.

Then, as the most important contribution, I specified if
such behavior is in fact drived by ``players' ''
reference-dependent preference. In other words, I try to answer
a question ``Are there any monetary incentives that make players
turn to .300 of batting-average? `` If MLB team managers overestimate
the contribution of the players with .300 of batting-average to
raising winning-perentage of the team and pay them discontinuously
higher salary, then the observed behavior of the players are
economically rational.

\section{Theoretical Frameworks and Literature Review}

 \subsection{Literature}

  Tversky and Kahneman (1992) mentioned reference point dependence
  as one of the two distinct respects of their prospect theory.
  The most primitive form of reference dependent utility function is:
  \[
  u(x | r) = \begin{cases}
  x - r & \text{ if }x \geq r \\
  \lambda (x - r) & \text{ if }x < r
\end{cases}
  \]

  where $x$ denotes a certain outcome, and $r$ is one of
  reference point. This agent evaluates the outcome by the
  difference from the reference point. In adiition, they
  addumed that agents regard loss as larger than gain,
  or ``loss aversion." Then, $\lambda$ is restricted to larger than 1.

  There are a number of empirical literature that specifies the existence
  of reference dependence in the field or lab studies.

  Among them, the most important literature for this paper is Pope and
  Simonsohn (2011). They picked up three empirical evidence of round number
  as reference points: SAT (a standardized test for college admission in the
  United States) scores, laboratory experiment, and baseball.
  In their section of baseball, they picked up the evidence of
  Major League Baseball (MLB) players and pointed out that they pay attention
  to their batting-average (AVG), especially to whether they could finish
  each season with their batting average of just above .300.
  They observed position players (= players except for pitchers) with at
  least 200 at-bats in each season, and found that their distribution
  of the batting-average has excess mass just above .300.
  Furthermore, they found that palyers with batting-average of just below
  .300 are more likely to hit a base-hit and less likely to get a
  base-on-balls.
  Both base-hits and base-on-balls avoid the batter from out, so for the
  team he belongs to, base-on-balls also have important value to win the
  game. However, batting-average does not count base-on-balls as the
  element to raise the number (For the definition of performance
  statistics, see Appendix). Thus, observed behavior they claims is
  sufficient evidence that shows the existance of round-number reference
  point deendent preference.



 \subsection{Frameworks}

 Excess mass around the reference point dependence is caused by
 ``notch'' in the utility function. In this paper, I exploit the
 framework to specify the utility function of Diecidue and
 Van De Ven (2008)`s ``aspiration level" model, following the
 way

 \[
 \lim_{\epsilon \to 0} v_r (r + \epsilon) \neq
 \lim_{\epsilon \to 0} v_r (r - \epsilon)
 \]

 This form of utility function is discontinuous at the
 reference point $r$.


 \subsection{Impact of `` \textit{Moneyball} ``
 - Time Series Analysis -}


\section{Method and Data}

 \subsection{Empirical Method}

  \subsubsection{Test for Bunching}

  \subsubsection{Monetary Incentive}

  I examine the existance of the monetary incentive by regression
  analysis below:

  \[
  w_{it} = \beta_0 & X_{it} + \beta_1 \text{ABOVE}_{it}
  + \beta_2 X_{it} \times \text{ABOVE}_{it} + \beta_3 Z_{it}
  \]

  For each player $i$ in the season $t$, $w_{it}$ is log annual salary
  in next season $t+1$. $X_{it}$ and $Z_{it}$ are the value of performance
  statistics (batting-average, on-base percentage,\ldots) and other
  player-specific characters (age, team he signed, position,\ldots),
  respectively. $\text{ABOVE}_{it}$ is a dummy variable that takes 1
  if the player achieves their reference point, and 0 if the failed
  to do so. I also introduce interaction term of $X_{it}$ and
  $\text{ABOVE}_{it}$. Age term is introduce as quadratic form.
  I utilize both simple OLS and player, team, positional fixed effect model.



 \subsection{Data Description}

 In order to make empirical research, I need information
 about players' performance, contracts and other details.
 Then, I generated panel data that contains these specific information
 from some open data-source. Each sample is obtained by unit of a
 single season, but due to lack of open source of information about countracts,
 time range of the data used in each analysis is unbalanced.

 Performance statistics is obtained from baseball fan website:
 \textit{fangraphs} and \textit{Baseball Reference}.





\section{Result}

 \subsection{Full-Sample analysis}
  \subsubsection{Existance of Excess Mass around Round Number}



 \subsection{Time-Series analysis}

 Through the previous section, I conducted analysis including
 all the sample as same, as Pope and Simonsohn (2011) did.
 However, there is some time-specific factor that affect on
 the behavior of the players, such as the relationship
 between the player and the team manager, what index
 was of the most importance to evaluate players, or some
 macroeconomic trend.



\section{Discussions and Conclusion}



\section{Appendix}

\subsection{Proxies for Performance}

To measure batter's performance, I use some \textit{SABR
Metrics}'s statistics. \textit{SABR} is short for \textit{Society
of American Baseball Research}, an American organization of
analysing baseball. They try to evaluate players in more ``efficient ''
way,  or by statistics with more close correlation with the
expected wins the team obtain.

\section{References}
\small
\begin{itemize}
  \item
\end{itemize}

\end{document}
